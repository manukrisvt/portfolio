\documentclass{article}
\usepackage{scimisc-cv}
\usepackage{etoolbox,refcount}
\usepackage{multicol}

\newcounter{countitems}
\newcounter{nextitemizecount}
\newcommand{\setupcountitems}{%
  \stepcounter{nextitemizecount}%
  \setcounter{countitems}{0}%
  \preto\item{\stepcounter{countitems}}%
}
\makeatletter
\newcommand{\computecountitems}{%
  \edef\@currentlabel{\number\c@countitems}%
  \label{countitems@\number\numexpr\value{nextitemizecount}-1\relax}%
}
\newcommand{\nextitemizecount}{%
  \getrefnumber{countitems@\number\c@nextitemizecount}%
}
\newcommand{\previtemizecount}{%
  \getrefnumber{countitems@\number\numexpr\value{nextitemizecount}-1\relax}%
}
\makeatother    
\newenvironment{AutoMultiColItemize}{%
\ifnumcomp{\nextitemizecount}{>}{3}{\begin{multicols}{2}}{}%
\setupcountitems\begin{itemize}}%
{\end{itemize}%
\unskip\computecountitems\ifnumcomp{\previtemizecount}{>}{3}{\end{multicols}}{}}

\title{Scismic's Recommended CV for Biotech and Pharma Positions}
\author{Scismic: The Talent Matching Platform for The Life Sciences (www.scismic.com)}
\date{May 2020}

%% These are custom commands defined in scimisc-cv.sty
\cvname{Manu Krishnan, Ph.D}
\cvpersonalinfo{
Charlotte, NC \cvinfosep 
540-449-7532 \cvinfosep
manukrishnantvm@gmail.com \cvinfosep
}

\begin{document}

% \maketitle %% This is LaTeX's default title constructed from \title,\author,\date

\makecvtitle

\section{Summary: Strategic AI Leader \& Sr. Data Scientist}
Results-driven Sr. Data Scientist and technical leader with $7+$ years spanning predictive health monitoring, AI innovation, and academic research. Demonstrated expertise driving enterprise analytics, leading cross-functional teams, and developing production-grade AI/ML solutions to optimize complex system reliability. Adept at translating business needs into technical product strategies and driving organizational outcomes via data-driven innovation, executive collaboration, and agile project management.

\noindent\textbf{Core Competencies:}
\begin{itemize}
    \item Predictive Modeling \& Prognostics Algorithms
    \item Enterprise AI Strategy, Predictive Analytics, and ML
    \item Structural Health Monitoring \& Diagnostics
    \item Cross-functional Team Leadership \& Agile Project Management
    \item AI Agents (Google ADK, LangGraph), LLMs, Retrieval-Augmented Generation (RAG)
    \item Big Data Processing (Spark, Delta Lake, MLOps)
    \item Git Version Control and Software Product Lifecycle
    \item Time Series AI and IMU Sensor Data Modeling
\end{itemize}

\section{Technical Skills}
\begin{itemize}
\begin{AutoMultiColItemize}
    \item \textbf{Programming:} Python, Java, R, $C++$, Matlab
    \item \textbf{Big Data/Cloud:} Databricks, Spark, Delta Lake, Git, MLOps workflows
    \item \textbf{GenAI:} LLMs, RAG, LangChain, LangGraph, Google ADK
    \item \textbf{Data Engineering:} SQL, workflow automation
    \item \textbf{Visualization/Reporting:} Power BI, Tableau, Excel, PowerPoint
    \item \textbf{Modeling/Test:} Ansys, Abaqus, Nastran, FEMap, Labview
\end{AutoMultiColItemize}
\end{itemize}

\section{Work Experience}

\cvsubsection{Joby Aviation}[Santa Cruz, CA]
[Sr. Data Scientist (Health Usage and Monitoring)][Jan 2022 -- Present]
\begin{itemize}
    \item Spearheaded predictive health and monitoring for aviation systems using ML/AI, deploying $5+$ models for structural diagnostics and maintenance optimization.
    \item Led team to build AI agents (LLMs + RAG) for production fleet analysis; reduced expert workload by $40\%$ and improved maintenance responsiveness.
    \item Improved data pipeline runtimes by $60\%$, integrating Databricks-based solutions for real-time risk detection and model validation.
    \item Developed and validated time-series machine learning models using IMU sensor data for vibration analysis and reliability prediction of advanced aircraft assets.
    \item Drove collaboration with executives/FAA for product certification and business impact; established rigorous adoption and KPI tracking.
\end{itemize}

\cvsubsection{Joby Aviation}[Santa Cruz, CA]
[Propeller Integrity Intern][May 2021 -- Aug 2021]
\begin{itemize}
    \item Developed real-time machine learning toolkits (Python/Databricks) for structural imbalance and failure detection.
    \item Modeled operational bearing harmonics and designed monitoring protocols deployed across critical fleet assets.
    \item Applied advanced feature engineering with IMU time series data for anomaly detection and vibration monitoring in propeller subsystems.
\end{itemize}

\section{Research Experience}

\cvsubsection{Virginia Tech}[Blacksburg, VA]
[Graduate Research Assistant][Sept 2017 -- Dec 2021]
\begin{itemize}
    \item \textbf{PhD Dissertation:} Dynamic data-driven modeling of vibration in aircraft engine
    \begin{itemize}
        \item Led PhD research on dynamic machine learning models for structural vibration and health monitoring, leveraging IMU sensor and time series data for aerospace reliability applications.
        \item Developed and validated multiphysics, time series ML models; mentored undergraduate researchers and partnered with sponsors for technology transfer.
        \item Produced 2 high-impact publications and filed a patent based on novel AI-driven vibration diagnostics.
    \end{itemize}
\end{itemize}

%% An example of leaving an argument empty
\cvsubsection{Indian Institute of Technology (IIT) - Guwahati}[Guwahati, India][Graduate research][Jan 2016 to May 2017]

\begin{itemize}
    \item Developed recursive PCA and AR-based real-time detection algorithms; published in leading journals.
\end{itemize}

\section{Education}
\cvsubsection{Virginia Tech}[Blacksburg, VA]
[\textbf{PhD (STEM) - Aerospace Engineering (Structures)} - Current GPA 3.96][Sept 2017 to Dec 2021]
\begin{itemize}
\item Elastic Stability, Advanced Aero hydrodynamics, Structural Optimization, Vehicle Structures, Dynamical systems and controls.
\item \textbf{Graduate certificate in Data analytics} - Data analysis - I, Bayesian analysis, Time series analysis, Advanced Machine learning.
\end{itemize}
\cvsubsection{Indian Institute of Technology (IIT) - Guwahati}[Guwahati, India][M. tech - Structural Engineering - GPA 4.0 (Batch topper)][Sept 2015 to May 2017]
\begin{itemize}
\item Structural analysis, Structural dynamics, Finite element methods, Advanced Structural system Design, Reliability based design.
\end{itemize}

%--------AWARDS----------
\section{Honors \& Fellowships}
\begin{itemize}
    \item John R. Jones III Graduate Fellowship -- Virginia Tech
    \item Rolls Royce Fellowship -- Virginia Tech / Rolls Royce
    \item Pratt Fellowship, Structural Engineering Batch Topper (IIT-G)
\end{itemize}

%--------PROFESSIONAL SOCIETY----------
\section{Professional Membership}
\begin{itemize}
    \item SAE HM-1 Integrated Vehicle Health Management Liaison
    \item Society of Experimental Mechanics (SEM)
    \item American Institute of Aeronautics and Astronautics (AIAA)
\end{itemize}

 \section{Journal Publications}
 \begin{itemize}
     \item \textbf{Krishnan, M}, Bhowmik, B., Hazra, B., and Pakrashi, V. (2018).  Real time damage detectionusing recursive principal components and time varying auto-regressive modeling. Mechanical Systems and Signal Processing, 101:549–574.
     \item \textbf{Krishnan, M}, Bhowmik, B., Tiwari, A., and Hazra, B. (2017). Online damage detection using recursive principal component analysis and recursive condition indicators. Smart Materials and Structures, 26(8):085017
     \item Bhowmik, B.,\textbf{Manu Krishnan}, Hazra, B., and Pakrashi, V. (2019).  Real-time unified single-and multi-channel structural damage detection using recursive singular spectrum analysis. Structural Health Monitoring, 18(2):563–589.
     \item Malladi, V.V.S., Albakri, M.I., \textbf{Krishnan, M.}, Gugercin, S. and Tarazaga, P.A. (2022). Estimating experimental dispersion curves from steady-state frequency response measurements. Mechanical Systems and Signal Processing, 164, p.108218.
     \item \textbf{Krishnan, M.}, Sever, I.A. and Tarazaga, P. (2022). Data-Driven Modeling of Vibrations in Turbofan Engines Under Different Operating Conditions. AIAA Journal, pp.1-15.
     \item \textbf{Krishnan, M.}, Malladi, V.V.S. and Tarazaga, P.A. (2022). Leveraging a data-driven approach to simulate and experimentally validate a MIMO multiphysics vibroacoustic system. Mechanical Systems and Signal Processing, 166, p.108414.
     \item \textbf{Krishnan, M.}, Gugercin, S., and Tarazaga, P. A. (2023). A wavelet-based dynamic mode decomposition for modeling mechanical systems from partial observations. Mechanical Systems and Signal Processing, 187, 109919.
 \end{itemize}
 \section{Conference Publications}
 \begin{itemize}
     \item Krishnan, M., Gugercin, S., Sever, I., and Tarazaga, P. (2020). Dynamic Data Driven Modeling of Aero Engine Response. In Model Validation and Uncertainty Quantification, Volume 3: Proceedings of the 38th IMAC, A Conference and Exposition on Structural Dynamics 2020 (pp. 273-278). Springer International Publishing.
\item Krishnan, M., Jin, R., Sever, I. A., and Tarazaga, P. A. (2020). Data based modeling of aero engine vibration responses. In Sensors and Instrumentation, Aircraft/Aerospace, Energy Harvesting & Dynamic Environments Testing, Volume 7: Proceedings of the 37th IMAC, A Conference and Exposition on Structural Dynamics 2019 (pp. 365-368). Springer International Publishing.
\item Krishnan, M., Gugercin, S., and Tarazaga, P. (2021). Wavelet‐based dynamic mode decomposition. PAMM, 20(S1), e202000355.
\item Krishnan, M., Sever, I. A., and Tarazaga, P. A. (2020). Determining Interdependencies and Causation of Vibration in Aero Engines Using Multiscale Cross-Correlation Analysis. In Model Validation and Uncertainty Quantification, Volume 3: Proceedings of the 38th IMAC, A Conference and Exposition on Structural Dynamics 2020 (pp. 265-272). Springer International Publishing.
\item Bhowmik, B., Krishnan, M., Hazra, B., and Pakrashi, V. (2017). Online damage detection using recursive principal component analysis. Procedia engineering, 199, 2108-2113.
\item Davaria, S., Krishnan, M., Sriram Malladi, V. V., and Tarazaga, P. A. (2022, August). Miniature Underwater Robot–An Experimental Case Study. In Special Topics in Structural Dynamics & Experimental Techniques, Volume 5: Proceedings of the 40th IMAC, A Conference and Exposition on Structural Dynamics 2022 (pp. 119-122). Cham: Springer International Publishing.

 \end{itemize}


%  \cvsubsection{Virginia Tech}[Blacksburg, VA]
% [\textbf{Graduate certificate in Data analytics}][Jan 2018 to present]
% \begin{itemize}
% \item Data analysis - I, Bayesian analysis, Time series analysis, Advanced Machine learning. 
% \end{itemize}

\section{Scholarship and Awards}
\begin{itemize}
\item John R. Jones III Graduate student Fellowship - Offered by Department of Mechanical Engineering - Virginia Tech (2020 - present)
\item Rolls Royce Graduate Fellowship - Offered by Virginia Tech and Rolls Royce towards successful completion of PhD covering tuition expenses, stipend and consumables (2019 - present). 
\item Pratt Fellowship - Offered by Department of Aerospace and Ocean Engineering - Virginia Tech (2017-2018)
\item Structural Engineering 2015-17 Batch Topper - Offered by Indian Institute of Technology, Guwahati. 
\end{itemize}
\section{Professional Society Membership}
\begin{itemize}
\item Liaison to the SAE HM-1 Integrated Vehicle Health Management (2022 - Present)
\item Society of Experimental Mechanics (2017 - Present)
\item American Institute of Aeronautics and Astronautics (2017-Present)
\end{itemize}

% 



% Candidate open to obtaining security clearance if required


\end{document}
